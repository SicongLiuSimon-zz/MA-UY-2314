\documentclass[12pt]{article}
\usepackage[margin=1in]{geometry}
\usepackage{amsmath, amssymb, amsthm, graphicx, hyperref}
\usepackage{enumerate}
\usepackage{fancyhdr}
\usepackage{multirow, multicol}
\usepackage{tikz}
\pagestyle{fancy}
\fancyhead[RO]{Sicong Liu}
\fancyhead[LO]{MA-UY 2314: Discrete Mathematics}
\usepackage{comment}
\newif\ifshow
\showfalse
\begin{document}

\begin{center}
\textbf{\Large Homework 4}\\
\end{center}
\hrule

\vspace{0.2cm}
\begin{enumerate}
\item
\noindent 11.2(a) There is an integer that is not prime, $\exists x \in \mathbb{Z}$; x is not prime\\
\noindent 11.2(b) All integers are either prime or composite, $\forall x \in \mathbb{Z}$; x is either prime or composite\\
\noindent 11.2(c) All integers have a square that is not 2, $\forall x \in \mathbb{Z}$; $x^{2} \neq 2$\\
\noindent 11.2(d) There is an integer that is not divisible by 5, $\exists x \in \mathbb{Z}$; $5|x =F$\\
\noindent 11.2(e) All integers are not divisible by 7,  $\forall x \in \mathbb{Z}$; $7|x =F$\\
\noindent 11.2(f) There is an integer that has a square that is negative, $\exists x \in \mathbb{Z}$; $x^{2} <0$\\
\noindent 11.2(g) There is an integer that the products with any integers are not 1, \\$\exists x \in \mathbb{Z}$;$\forall y \in \mathbb{Z}$; $xy \neq 1$\\
\noindent 11.2(h) For any two integers, the quotient of them is not equal to 10, \\$\forall x \in \mathbb{Z}$;$\forall y \in \mathbb{Z}$; $x/y \neq 10$\\
\noindent 11.2(i) For any integers, there exists another integer that the product of them is not 0,\\ $\forall x \in \mathbb{Z}$;$\exsits y \in \mathbb{Z}$; $xy \neq 0$\\
\noindent 11.2(j) There is an integer that any integers are smaller or equal to it. \\$\exists x \in \mathbb{Z}$;$\forall y \in \mathbb{Z}$; $y \leq x$\\
\noindent 11.2(k) Sombody does not love everybody, \\$exists \; a \;person\; x$;$\forall \;person\; y$; x does not love y\\
\noindent 11.4(a) F (b) T (c) F (d) T (e) F (f) T (g) T (h) T\\

\item 
\noindent 11.5(a) $\exists x \in \mathbb{Z}$; $x \geq 0$\\ There is an integer that is larger than or equal to 0\\
\noindent 11.5(b) $\forall x \in \mathbb{Z}$; $x \neq x+1$\\ All integers are not equal to the next integer\\
\noindent 11.5(c) $\forall x \in \mathbb{N}$; $x \leq 10$\\ All national numbers are less than or equal to 10\\
\noindent 11.5(d) $\exists x \in \mathbb{N}$; $x+x \neq 2x$\\ There is a national number which the sum of it with itself is not equal to the twice of it\\
\noindent 11.5(e) $\forall x \in \mathbb{Z}$; $\exists y \in \mathbb{Z}$;$x \leq y$\\ For all integers, there exists an integer less than or equal to it\\
\noindent 11.5(f) $\exists x \in \mathbb{Z}$; $\exists y \in \mathbb{Z}$; $x \neq y$\\ There exist an integer that is not equal to another integer\\
\noindent 11.5(g)  $\exists x \in \mathbb{Z}$; $\forall y \in \mathbb{Z}$; $x+y \neq 0$\\ There exist an integer that the sum of it and any other integer is zero\\
\noindent 11.6 Both of the statements are the same thing. The order of them does not matter.
\noindent 11.7(a) True, x can only be 2\\
\noindent 11.7(b) False, x can be 2 or -2\\
\noindent 11.7(c) True, x can only be $\sqrt{3}$\\
\noindent 11.7(d) True, x can only be 0\\
\noindent 11.7(e) True, x can only be 1\\

\item
\noindent 12.1(a) $\{1,2,3,4,5,6,7\}$\\
\noindent 12.1(b) $\{4,5\}$\\
\noindent 12.1(c) $\{1,2,3\}$\\
\noindent 12.1(d) $\{6,7\}$\\
\noindent 12.1(e) $\{1,2,3,6,7\}$\\
\noindent 12.1(f) $\{(1,4),(1,5),(1,6),(1,7),(2,4),(2,5),(2,6),(2,7),(3,4),(3,5),(3,6),(3,7),(4,4),(4,5),(4,6),(4,7),(5,4),(5,5),(5,6),(5,7)\}$\\
\noindent 12.1(g) $\{(4,1),(4,2),(4,3),(4,4),(4,5),(5,1),(5,2),(5,3),(5,4),(5,5),(6,1),(6,2),(6,3),(6,4),(6,5),(7,1),(7,2),(7,3),(7,4),(7,5)\}$\\

\item
\noindent \textbf{Proof:}  Assume that $x \in (A \cup B) - (A \cap B)$. \\
NTS $x \in A \triangle B$\\
$\to (x \in A \vee x \in B) \wedge (x \notin A \vee x\notin B)$\\
$\to (x \in A) \wedge (x \notin A \vee x\notin B)) \vee (x \in B) \wedge (x \notin A \vee x\notin B)$\\
$\to ((x \in A \wedge x \notin A) \vee (x\in A \wedge x\notin B)) \vee ((x \in B \wedge x \notin B) \vee (x\in B \wedge x\notin A))$\\
$\to (F \vee (x\in A \wedge x\notin B)) \vee (F \vee (x\in B \wedge x\notin A))$\\
$\to (x\in A \wedge x\notin B) \vee (x\in B \wedge x\notin A)$\\
$\to (x \in A-B) \vee (x \in B-A)$\\
$\to x \in A \triangle B$\\

\item
\noindent 12.14 To prove $A - \emptyset = A$
\textbf{\begin{enumerate}[(a).]
    \item \textit{$x \in A - \emptyset \longrightarrow x \in A$}
    \item \textit{$x \in A \longrightarrow x \in A - \emptyset$}
\end{enumerate}}
\noindent \textbf{Proof of (a):}  Assume that $x \in A - \emptyset$.\\
$\to x\in A \wedge x\notin \emptyset$\\
$\to x\in A$\\
\textbf{\noindent \textbf{Proof of (b):}  Assume that $x \in A$.}\\
NTS $x\in A \wedge x\notin \emptyset$\\
$\to x\notin \emptyset$\\
$\to x\in A \wedge x\notin \emptyset$\\
To prove $\emptyset - A = \emptyset$
$\emptyset - A \subseteq \emptyset \wedge \emptyset \subseteq \emptyset - A$\\
$\to \emptyset - A = \emptyset $\\

\vspace{0.15in}
\noindent 12.15 To prove $A \triangle A= \emptyset$\\
\textbf{\begin{enumerate}[(a).]
    \item \textit{$x \in A \triangle A \longrightarrow x \in \emptyset$}
    \item \textit{$x \in \emptyset \longrightarrow x \in A \triangle A$}
\end{enumerate}}
\noindent \textbf{Proof of (a):}  Assume that $x \in A \triangle A$.\\
$\to x \in A-A$\\
$\to x \in \emptyset$\\
\noindent \textbf{Proof of (b):}  Assume that $x \in \emptyset$.\\
$\to x \in A-A$\\
$\to x \in A \triangle A$\\
\vspace{0.15in}
To prove $A \triangle \emptyset= A$
\textbf{\begin{enumerate}[(a).]
    \item \textit{$x \in A \triangle \emptyset \longrightarrow x \in A$}
    \item \textit{$x \in A \longrightarrow x \in A \triangle \emptyset$}
\end{enumerate}}
\noindent \textbf{Proof of (a):}  Assume that $x \in A \triangle \emptyset$.\\
$\to x \in (A-\emptyset) \cup (\emptyset-A)$\\
$\to x \in A\cup \emptyset$\\
$\to x \in A$\\
\noindent \textbf{Proof of (b):}  Assume that $x \in A$.\\
NTS $x \in (A-\emptyset) \cup (\emptyset-A)$\\
$A=A\cup \emptyset$
$\to x \in A \cup \emptyset$\\
$\to x \in A \triangle \emptyset$\\

\item
\noindent 12.21(a) F $A={1,2,3},B={2},C={3},A-(B-C)={1,3},(A-B)-C={1}$\\
\noindent 12.21(b) T \\
\noindent 12.21(c) F $A={1,2,3},B={2},C={3},A-(B-C)={1,3}, (A \cup B)-C={1,2}, (A-C) \cap (B-C)={2}$\\
\noindent 12.21(d) F $A={1,2,3},B={1,2,3},C={4},B-C={1,2,3}=A,A \cup C={1,2,3,4} \neq B$\\
\noindent 12.21(e) F $A={1,2,3},B={1,2,3},C={3},A \cup C ={1,2,3}=B,B-C={1,2} \neq A$\\
\noindent 12.21(f) F $A={1,2},B={2,3}, |A-B|=1, |A|-|B|=0$\\
\noindent 12.21(g) F $A={1,2},B={2,3}, A-B={1}, (A-B) \cup B={1,2,3} \neq A$\\
\noindent 12.21(h) F $A={1,2},B={2,3}, (A \cup B)-B={1} \neq A$\\

\item
\noindent \textbf{Proof of (a):} Assume that $x \in A - (B \cup C).$\\
NTS $x \in (A - B) \cap (A - C)$\\
$\to x \in A \wedge x \notin (B \cup C)$\\
$\to x \in A \wedge x \notin B \wedge x\notin C$\\
$\to (x \in A \wedge x \notin B) \wedge (x \in A \wedge x \notin C$\\
$\to x \in (A-B) \wedge X \in (A-C)$\\
$\to x \in (A - B) \cap (A - C)$\\
\vspace{0.15in}
\noindent \textbf{Proof of (b):} Assume that $x \in (A - B) \cap (A - C)$\\
NTS $x \in A - (B \cup C).$
$\to x \in (A-B) \wedge X \in (A-C)$\\
$\to (x \in A \wedge x \notin B) \wedge (x \in A \wedge x \notin C)$\\
$\to x \in A \wedge (x \notin B \wedge x\notin C)$\\
$\to x \in A \wedge x \notin (B \cup C)$\\
$\to x \in A - (B \cup C)$\\

\item
Let $D=A \cup B$\\
LHS=$|Z \cup C|$\\
$|Z \cup C|=|Z|+|C|-|Z \cap C|$\\
$=|A \cup B|+|C|-|A\cup B|\cap C$\\
$=|A|+|B|-|A\cap B|+|C|-|(A\cap C)\cup(B \cap C)|$\\
$=|A|+|B|-|A\cap B|+|C|-|A\cap C|-|B\cap C|+|(A\cap C)\cap(A\cap B)|$\\
$=|A|+|B|+|C|-|A\cap B|-|A\cap C|-|B\cap C|+|A\cap B\cap C|=RHS$\\

\item
\noindent 14.1(a) $(1,2),(1,3),(1,4),(1,5),(2,3),(2,4),(2,5),(3,4),(3,5),(4,5)$\\
\noindent 14.1(b) $(1,1),(1,2),(1,3),(1,4),(1,5),(2,2),(2,4),(3,3),(4,4),(5,5)$\\
\noindent 14.1(c) $(1,1),(2,2),(3,3),(4,4),(5,5)$\\
\noindent 14.1(d) $(1,1),(1,3),(1,5),(2,2),(2,4),(3,1),(3,3),(3,5),(4,2),(4,4),(5,1),(5,3),(5,5)$\\
\noindent 14.2(a) For $a,b \in \{1,2,3,4,5\}, (a,b) \in R$ iff $b-a=1$\\
\noindent 14.2(b) For $a,b \in \{1,2,3,4,5\}, (a,b) \in R$ iff $a \geq b$\\
\noindent 14.2(c) For $a,b \in \{1,2,3,4,5\}, (a,b) \in R$ iff $a+b=6$\\
\noindent 14.2(d) For $a,b \in \{1,2,3,4,5\}, (a,b) \in R$ iff $a|b$\\

\item
\noindent 14.3(a) reflexive, symmetric, antisymmetric, transitive\\
\vspace{0.05in}
For irreflexive, x=1,1 R 1=T\\
\noindent 14.3(b) irreflexive, antisymmetric\\
For reflexive, x=1, 1 R 1=F\\
For symmetric, $(1,2)\in R, (2,1) \notin R$\\
\vspace{0.05in}
For transitive, $(1,2),(2,3) \in R, (2,3) \notin R$\\
\noindent 14.3(c) antisymmetric, transitive\\
For reflexive, x=2, 2 R 2=F\\
For irreflexive, x=1,1 R 1=T\\
\vspace{0.05in}
For symmetric, $(1,2)\in R, (2,1) \notin R$\\
\noindent 14.3(d) symmetric\\
For reflexive, x=2, 2 R 2=F\\
For irreflexive, x=1,1 R 1=T\\
For antisymmetric, $(1,2)\in R, (2,1) \in R,1 \neq 2$\\
\vspace{0.05in}
For transitive, $(1,1),(1,2) \in R, (1,2) \notin R$\\
\noindent 14.3(e) reflexive, symmetric, transitive\\
For irreflexive, x=1,1 R 1=T\\
\vspace{0.05in}
For antisymmetric, $(1,2)\in R, (2,1) \in R,1 \neq 2$\\
\noindent 14.4(a) reflexive, symmetric, transitive\\
\noindent 14.4(b) irreflexive, antisymmetric\\
\noindent 14.4(c) reflexive, symmetric, transitive\\
\noindent 14.4(d) reflexive, symmetric\\
\noindent 14.4(e) irreflexive, symmetric\\
\noindent 14.4(f) irreflexive, antisymmetric\\

\end{enumerate}
\end{document}