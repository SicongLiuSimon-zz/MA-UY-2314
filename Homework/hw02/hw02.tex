\documentclass[12pt]{article}
\usepackage[margin=1in]{geometry}
\usepackage{amsmath, amssymb, amsthm, graphicx, hyperref}
\usepackage{enumerate}
\usepackage{fancyhdr}
\usepackage{multirow, multicol}
\usepackage{tikz}
\pagestyle{fancy}
\fancyhead[RO]{Sicong Liu}
\fancyhead[LO]{MA-UY 2314: Discrete Mathematics}
\usepackage{comment}
\begin{document}
\begin{center}
  \textbf{\Large Homework 2}\\
  \vspace{.15in}
\fi
\end{center}
\noindent 1.(a) 
\noindent \textbf{Pf:} Assume $x$ and $y$ are consecutive perfect squares.\\
Counterexample: \\
$x=0$, $x=a^{2}$, $a=0$\\
$y=(a+1)^{2} \to 1^{2}$\\
$\to y=1$\\
\vspace{.15in}
$x-y=0-1=-1 \to -1$ is not even\\ 
(b)
\noindent \textbf{Pf:}  Assume $x$ and $y$ are consecutive perfect squares. $x=a^{2}, y=(a+1)^&{2}$\\
NTS $x-y=2k-1 $ where $ k\in \mathbb{Z}$
\[
\begin{aligned}
x-y&=a^{2}-(a+1)^{2}\\
&=a^{2}-a^{2}-1-2a\\
&=-2a-1\\
&=2k-1 \; when \; k=-a\\
\vspace{.15in}
\end{aligned}
\]
2. 
\noindent \textbf{Pf:}  Assume $a|d$ and $a |e$.
$d=ma, e=na$ where $ m\in \mathbb{Z} \wedge$, where $ n\in \mathbb{Z}$ \\
NTS $ db + ec=ka$ where $ k\in \mathbb{Z}$\\ 
\[
\begin{aligned}
db+ec&=ma \times b+na \times c\\
&=a \times mb + a \time nc\\
&=a(mb+nc)\\
&=ka \; when \; k=mb+nc\\
\vspace{.15in}
\end{aligned}
\]
3. 
\noindent \textbf{Pf:} Assume $a$ is an integer.\\
NTS $ 2 | (a^{2}-a) $\\ 
$a^{2}-a&=a(a-1)$\\
case 1:if a is even $\to$ $a=2m$, where $m\in \mathbb{Z}$\\
$a^{2}-a=2m(2m-1)$
$\to 2 | 2m(2m-1)$\\
$\to 2 | (a^{2}-a)$, when a is even\\
case 2:if a is odd $\to$ $a=2n+1$, where $n\in \mathbb{Z}$\\
$a^{2}-a=(2n+1)(2n+1-1)=2n(2n+1)$
$\to 2 | 2n(2n+1)$\\
$\to 2 | (a^{2}-a)$, when a is odd\\
\vspace{.15in}
Therefore, $ 2 | (a^{2}-a) $ \\


\noindent 4.
counterexample: \\
$x = F, y = T$\\
$x \to y= F \to T = T, -y \to -x= F \to T$\\
\vspace{.15in}
$x \leftrightarrow y = F \leftrightarrow T = F$\\


\noindent  5.
\noindent \textbf{Pf:} Assume that $x$ and $y$ are integers with the same parity.\\
NTS $x+y=2k$ where $k\in \mathbb{Z}$ 
when x and y are even\\
$\to$ $x=2m,y=2n$ where $m\in \mathbb{Z} \wedge n\in \mathbb{Z}$ \\
\[
\begin{aligned}
x+y&=2m+2n\\&=2(m+n)\\&=2k\; when\; k=m+n\\
\end{aligned}
\]

\noindent when x and y are odd\\
$\to$ $x=2m+1,y=2n+1$ where $m\in \mathbb{Z} \wedge n\in \mathbb{Z}$ \\
\[
\begin{aligned}
x+y&=(2m+1)+(2n+1)\\&=2m+2n+2\\&=2(m+n+1)\\&=2k\; when\; k=m+n+1\\
\end{aligned}
\]

\noindent 6.
\noindent \textbf{Pf:} Assume that $x, y, z, w$ are consecutive integers.\\
$\to x=a. y=a+1, z=a+2, w=a+3 $ where $a\in \mathbb{Z}$\\
NTS $x \times y \times \ z \time w = k^{2}-1$ where $k \in \mathbb{Z}$
\[
\begin{aligned}
x \times y \times \ z \time w &= a(a+1)(a+2)(a+3)\\
&=a(a+3)\times (a+1)(a+2)\\
&=(a^{2}+3a)(a^{2}+3a+2)\\
m=a^{2}+3a\\
m(m+2)&=(m+1)^{2}-1\\
&=k^{2}-1 \; where \;k=m+1
\end{aligned}
\]
\noindent 7.
\noindent \textbf{Pf:} Assume that $a,b,c$ are integers.\\
counterexample: $a=2, b=3, c=12$\\
\vspace{.15in}
$a|c, b|c, a+b=, 5|12=F \to (a+b)|c=F$

\noindent 8.
\noindent \textbf{Pf:} Assume that $n\in \mathbb{Z}$\\
counterexample: $n=5, n^{2}-n+1=25-5+1=21,$\\
\vspace{.15in}
$3|21, 21 is not prime$\\

\noindent 9.
\noindent \textbf{Pf:} Assume that $x$ is an odd integer. \\
$x=2k+1$ where $k \in \mathbb{Z}$\\
NTS $x^{2}=8m+1$\\
\[
\begin{aligned}
x^{2}&=(2k+1)^{2}\\&=4k^{2}+4k+1\\
\end{aligned}
\]
NTS $4k^{2}+4k+1=8m+1$\\
$4k^{2}+4k=8m$\\
$k^{2}+K=2m$\\
$k(k+1)=2m$\\
case 1: if k is even $\to k=2n$ where $n\in \mathbb{Z}$\\
$2n(2n+1)=4n^{2}+2n=2(n^{2}+n)=2m$ when $m=n^{2}+n$\\
case 2: if k is odd $\to k=2n+1$ where $n\in \mathbb{Z}$\\
$(2n+1)(2n+2)=4n^{2}+6n+2=2(n^{2}+3n+1)=2m$ when $m=n^{2}+3n+1$\\
Therefore, $x^{2}=8m+1$
\end{document}
