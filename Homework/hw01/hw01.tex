\documentclass[12pt]{article}
\usepackage[margin=1in]{geometry}
\usepackage{amsmath, amssymb, amsthm, graphicx, hyperref}
\usepackage{enumerate}
\usepackage{fancyhdr}
\usepackage{multirow, multicol}
\usepackage{tikz}
\pagestyle{fancy}
\fancyhead[RO]{Sicong Liu}
\fancyhead[LO]{MA-UY 2314: Discrete Mathematics}
\usepackage{comment}

\begin{document}

\noindent 3.1(a) $3|100$ = \textbf{F}.  There is no $k \in \mathbb{Z}$ such that $100 = 3k$.
\vspace{.15in}

\noindent 3.1(b) $3|99$= \textbf{T}.  Notice that $99 = 3k$ where $k = 33$.  
\vspace{.15in}

\noindent 3.1(c) $3|-3$ = \textbf{T}.  Notice that $-3 = 3k$ where $k=-1$.
\vspace{.15in}

\noindent 3.1(d) $-5|-5$ = \textbf{T}.  Notice that $-5 = -5k$ where $k=1$.
\vspace{.15in}

\noindent 3.1(e) $-2|-7$ = \textbf{F}.  There is no $k \in \mathbb{Z}$ such that $-7 = -2k$.
\vspace{.15in}

\noindent 3.1f) $0|4$ = \textbf{F}.  There is no $k \in \mathbb{Z}$ such that $4 = 0k$.
\vspace{.15in}

\noindent 3.1(g) $-4|-0$ = \textbf{T}.  Notice that $0 = 4k$ where $k=0$.
\vspace{.15in}

\noindent 3.1(h) $-0|-0$ = \textbf{T}.  Notice that $0 = 0k$ where $k \in \mathbb{Z}$.
\vspace{.3in}

\noindent 3.5 \\
Prove: Every integer is a rational number\\
Assume ($a \in \mathbb{Z}$) \\ 
NTS\;$a=kb$ where ($b \in \mathbb{Z}$)   $\wedge$ ($b \neq 0$)\\
Let $b=1$ such that ($b \in \mathbb{Z}$) \\
Notice that $a = ab$ where $b=1$ $ $\\
Disprove: All rational numbers are integers\\
Counterexample: $a=2$, $b=3$, $a/b\in \mathbb{Z}$ = \textbf{F} $ $\\
\vspace{.3in}

\noindent 3.12(e) $100$ has 9 positive divisors: 1,2,4,5,10,20,25,50,100.
\vspace{.15in}

\noindent 3.12(f) $1000000$ has 49 positive divisors.
\vspace{.15in}

\noindent 3.12(k) $1\times2\times3\times4\times5\times6\times7\times8$ has 96 positive divisors.
\vspace{.15in}

\noindent 3.12(l) 0 has infinite positive divisors.
\vspace{.15in}

\noindent 7.10(c) 
\[ 
\begin{tabular}{|c|c|c|c|c|c|c|c|} 
\hline   
x & y & x $\vee$ y & -x & -y & x $\wedge$ (-y) & (-x) $\wedge$ y & x $\wedge$ (-y) $\vee$  (-x) $\wedge$ y\\ 
\hline
T & T & T & F & F & F & F & F \\
\hline 
T & F & T & F & T & T & F & T \\
\hline
F & T & T & T & F & F & T & T\\
\hline
F & F & F & T & T & F & F & F\\
\hline
\end{tabular} 
\] 
The truth table of x $\vee$ y and x $\wedge$ (-y) $\vee$  (-x) $\wedge$ y is not the same\\
x $\vee$ y is not logically equivalent to x $\wedge$ (-y) $\vee$  (-x) $\wedge$ y
\vspace{.3in}

\noindent 7.5
\[ 
\begin{tabular}{|c|c|c|c|c|c|} 
\hline   
x & y & x $\leftrightarrow$ y & -x & -y & (-x) $\leftrightarrow$ (-y)\\
\hline
T & T & T & F & F & T\\
\hline 
T & F & F & F & T & F\\
\hline
F & T & F & T & F & F\\
\hline
F & F & T & T & T & T\\
\hline
\end{tabular} 
\] 

\vspace{.3in}

\noindent 7.8
\[ 
\begin{tabular}{|c|c|c|c|c|c|c|c|c|} 
\hline   
x & y & z & x $\vee$ y & x $\vee$ y $\to$ z & x $\to$ z & y $\to$ z & (x $\to$ z) $\wedge$ (y $\to$ z)\\
\hline
T & T & T & T & T & T & T & T \\
\hline 
T & F & F & T & F & F & T & F \\
\hline
F & T & F & T & F & T & F & F\\
\hline
F & F & T & F & T & T & T & T\\
\hline
F & T & T & F & T & T & T & T\\
\hline
T & F & T & T & T & T & T & T\\
\hline
T & T & F & T & F & F & F & F\\
\hline
F & F & F & F & T & T & T & T\\
\hline
\end{tabular} 
\] 

\vspace{.3in}
\noindent 7.11(a)
\[
\begin{aligned}
(x \vee y) \vee (x \vee (-y))\\
&=(x \vee x) \vee (y \vee (-y))\\
&=T \vee T\\
&=T \\
\end{aligned}
\]

\vspace{.15in}
\noindent 7.11(b) 
\[
\begin{aligned}
(x \wedge (x \rightarrow y)) \to y &= (x \wedge ((-x) \vee y)) \to y\\
&=(((-x) \wedge x) \vee (y \wedge x)) \to y\\
&=(F \vee (y \wedge x)) \to y\\
&=(y \wedge x) \to y\\
&=-(y \wedge x) \vee y\\
&=(-x) \vee ((-y) \vee y)\\
&=(-x) \vee T\\
&=T \\
\end{aligned}
\]

\vspace{.15in}
\noindent 7.11(h)
\[
\begin{aligned}
Let\; A=(x \to y) \;and\; B=(x \to -y)\\
(A \wedge B) \to (-x)
&=-(A \wedge B) \vee (-x)\\
&=((-A) \vee (-B)) \vee(-x)\\
&=(-A) \vee ((-B) \vee (-x)))\\
&=(-A) \vee ((x \to -y) \vee (-x))\\
&=(-A) \vee (-((-x) \vee (-y)) \vee (-x))\\
&=(-A) \vee (-(x \wedge y) \vee (-x)) \\
&=(-A) \vee (-x \vee x) \wedge (-x \vee y)\\
&=(-A) \vee (T \wedge (-x \vee y)\\
&=(-A) \vee A\\
&=T \\
\end{aligned}
\]
\vspace{.15in}
\noindent 7.13(a)
\[
\begin{aligned}
(x \vee y) \wedge (x \vee -y) \wedge -x\\
&=(x \vee y)\wedge( (x \wedge -x) \vee (-x \wedge -y))\\
&=(x \vee y)\wedge (F \vee (-x \wedge -y))\\
&=(x \vee y)\wedge (-x \wedge -y)\\
&=(x \wedge (-x \wedge -y)) \vee (y \wedge (-x \wedge -y))\\
&=(F\wedge y) \vee (F \wedge -x)\\
&=F \vee F\\
&=F  \\
\end{aligned}
\]
\vspace{.15in}
\noindent 7.13(c)
\[
\begin{aligned}
((x \to y) \wedge (-x \to -y)) \wedge -y
&=((x \to y) \wedge ((x \vee y) \wedge -y)\\
&=((x \to y) \wedge ((x \wedge -y) \vee (y\wedge -y))\\
&=((x \to y) \wedge ((x \wedge -y) \vee F)\\
&=((x \to y) \wedge (x \wedge -y)\\
&=((-x \vee y)\wedge x) \wedge -y\\
&=((-x \wedge x) \vee (y \wedge x)) \wedge -y\\
&=(F \vee (y \wedge x))\wedge -y\\
&=(y \wedge x) \wedge -y\\
&= (y \wedge -y) \wedge x\\
&= F \wedge x\\
&= F  \\
\end{aligned}
\]
\end{document}
